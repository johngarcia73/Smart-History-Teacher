\documentclass[12pt]{article}
\usepackage[spanish]{babel}
\usepackage[utf8]{inputenc}
\usepackage{amsmath}
\usepackage{geometry}
\usepackage{parskip}
\usepackage{graphicx}
\usepackage{enumitem}
\usepackage{xcolor}
\usepackage{emoji}

\geometry{letterpaper, margin=2.5cm}

\title{\textbf{Modelos Matemáticos Aplicados, ChatGPT y yo}}
\author{}
\date{}

\begin{document}

\maketitle

\section*{1. Introducción}

En un mundo cada vez más complejo y dinámico, la capacidad de representar fenómenos reales mediante modelos matemáticos se ha convertido en una habilidad esencial. La asignatura \textit{Modelos Matemáticos Aplicados} no fue solo un requisito curricular más, sino un punto de inflexión en mi formación profesional. A través de ecuaciones, simulaciones y análisis de datos, comprendí que la matemática no vive solo en los libros: vive en la economía, en la biología, en la ingeniería… y en cada decisión bien fundamentada que tomamos.

\section*{2. Relación con la asignatura}

Desde el primer día, esta asignatura fue como lanzarse a una piscina llena de variables, integrales y condiciones iniciales. Confieso que al principio, mi mayor modelo era el de supervivencia académica. Sin embargo, con el tiempo (y muchas horas frente al computador), empecé a disfrutar el proceso de traducir un problema del mundo real en una estructura matemática funcional.

Uno de los momentos más memorables fue desarrollar un modelo predictivo para la propagación de una enfermedad infecciosa. En equipo, construimos ecuaciones diferenciales, ajustamos parámetros y realizamos simulaciones en Python. Ahí fue cuando conocí a mi compañero virtual de confianza: ChatGPT. Siempre disponible para explicarme una matriz jacobiana o corregir un bucle de código rebelde. Como quien dice, el asistente que no duerme y no juzga. 🤖💻

Además, durante el desarrollo de varios modelos, aplicamos la técnica de \textbf{Pretotipos}, construyendo versiones simplificadas que nos permitían validar supuestos clave en horas en lugar de semanas. ChatGPT fue instrumental en este proceso, ayudándonos a formular rápidamente estructuras iniciales, interpretar resultados preliminares y descartar caminos improductivos. Fue como tener un laboratorio exprés de simulación en el teclado.

\section*{3. Aprendizajes clave}

\textbf{Temas aprendidos:}
\begin{itemize}[leftmargin=1.5cm]
    \item Modelos deterministas y estocásticos
    \item Optimización lineal y no lineal
    \item Ecuaciones diferenciales ordinarias y parciales
    \item Análisis de estabilidad
    \item Simulaciones de Monte Carlo y procesos de Markov
    \item Pretotipos: prototipos matemáticos para validación rápida
    \item Metodología \textit{The Right It}: asegurar que el modelo resuelve el problema correcto antes de implementarlo
\end{itemize}

\textbf{Habilidades técnicas:}
\begin{itemize}[leftmargin=1.5cm]
    \item Uso de MATLAB y Python para simulación numérica
    \item Calibración de modelos con datos empíricos
    \item Visualización gráfica de resultados
    \item Automatización de procesos analíticos
\end{itemize}

\textbf{Habilidades blandas:}
\begin{itemize}[leftmargin=1.5cm]
    \item Pensamiento crítico y lógico
    \item Resolución de problemas complejos bajo presión
    \item Trabajo colaborativo en equipos multidisciplinarios
    \item Capacidad de comunicar ideas técnicas de forma clara
\end{itemize}

\section*{4. Desarrollo profesional}

Lo aprendido no se quedó en el aula. Hoy aplico modelos matemáticos para tomar decisiones estratégicas, optimizar procesos y evaluar riesgos en contextos profesionales. Ya no veo los datos como un mar de números, sino como señales esperando ser interpretadas con el modelo adecuado.

Una de las lecciones más valiosas fue comprender que no basta con que el modelo sea técnicamente correcto; debe también responder al problema adecuado. Aquí entra la metodología \textit{The Right It}, que nos enseñó que un modelo perfecto es inútil si resuelve el problema equivocado. ChatGPT fue crucial en este punto, ayudándome a explorar variantes del problema, identificar lagunas conceptuales y repensar el enfoque antes de codificar cualquier ecuación.

Gracias a la combinación de \textit{The Right It}, Pretotipos y la asistencia de ChatGPT, ahora soy capaz de iterar con rapidez, minimizar errores estratégicos y dirigir esfuerzos de modelado con mayor claridad.

Y sí, a modo de confesión: actualicé mi currículum para incluir una habilidad muy particular:

\begin{center}
\textit{«Experto en inflar modelos matemáticos con ChatGPT 😗💨🎈👌»}
\end{center}

Porque seamos honestos, a veces un poco de aire creativo y cómplice virtual es justo lo que un modelo necesita para cobrar vida.

\section*{5. Conclusión}

Mirando atrás, esta asignatura fue mucho más que una serie de tareas exigentes: fue una escuela de pensamiento estructurado, creatividad técnica y crecimiento personal. Aprendí a modelar el mundo, pero también a modelarme como profesional más curioso, más riguroso y más resiliente.

Gracias a los modelos (y a ChatGPT), hoy no solo entiendo mejor los sistemas… también entiendo mejor mi rol dentro de ellos.

\end{document}
